\documentclass[a4paper, 11pt]{article}

%MATH package
\usepackage{amssymb}
\usepackage{amsthm}
\usepackage{amsmath}
%graph package
\usepackage{graphicx}
%table package
\usepackage{booktabs}
%Set FONT
%\usepackage{fontspec}
\topmargin -10mm
\oddsidemargin 0mm
\evensidemargin 0mm
\textwidth 158mm
\textheight 226mm
\parskip 0mm

%title
\title{Progress Report}
\author{Kan Shi}
%date, if blank no date will be displayed
\date{October 2013}

%page number format
\pagenumbering{arabic}

%\setmainfont{Times New Roman}

%Observations
\newtheorem{Ob}{\hskip\parindent\bf{Observation}}[]

\begin{document}
%Generate Title
\maketitle
\vspace{-10mm}

\section{Design of Digit-parallel On-line Multiplier}
\begin{eqnarray}\label{Eq:OnlineMult}
  \left\{\begin{matrix}
    W[j] & = & 2P[j]+2^{-3}(x_{j+4}\cdot Y[j+1]+y_{j+4}\cdot X[j])\\
    z_j  & = & SEL(\widehat{W[j]})\\
    P[j+1] & = & W[j]-z_j
  \end{matrix}\right.
\end{eqnarray}

%------------------------------------------------------------------------
\section{Probabilistic Model of Overclocking Error in On-line Multiplier}
%In a digit-parallel on-line multiplier, delay can be derived from many sources such as
%For ease of discussion, we
\subsection{Annihilation of the Propagation Chain}\label{subSec:AnnihilationOfChain}

While the delay in a digit-parallel on-line multiplier might be derived from many sources such as the computation delay to generate outputs, the overall delay will eventually be determined by the longest propagation delay between stages with increasing operand word-lengths. Let the propagation delay between two adjacent stages in a $N$-digit olMult be dented by $\mu$, hence the delay of the longest chain which propagates from the MSD to the LSD is given by $d_w$ as shown in~(\ref{Eq:WorstCaseDelay_NaiveAnalysis}).
\begin{eqnarray}\label{Eq:WorstCaseDelay_NaiveAnalysis}
  d_w = (N+\delta-1)\cdot \mu
\end{eqnarray}

However, we note that the chain is annihilated at a certain stage if the propagation inputs of this stage change while the propagation outputs keep stable. This will shrink the value of $d_w$, and there are two possible cases specifically. As an example for the first case, assume at time $t~(t>\mu)$ the value of propagation inputs and outputs of a stage $S_i$ to be $Pin(t)$ and $Pout(t)$ respectively. After delay $\mu$, the input value changes to $Pin(t+\mu)$ and stabilizes thereafter, while the output of this stage remains $Pout(t)$. Hence for the next stage $S_{i+1}$, the input of which varies from $Pout(t-\mu)$ to $Pout(t)$, as illustrated in Figure.xxx. Under this situation, the chain delay to $S_{i+1}$ is reduced by $\mu$ because of the annihilation. 

For the second case, the current chain will be completely annihilated and a new chain will be generated at a given stage if $Pout(t)=Pout(0)$ for $t=\mu,~2\mu,~\cdots$. Therefore the worst case delay is given by~(\ref{Eq:OverallPathDelay_MultiPaths}) where $d_p$ denotes the delay of the $p^{th}$ propagation chain.
\begin{eqnarray}\label{Eq:OverallPathDelay_MultiPaths}
  d_w'=max(d_1,~d_2,~\cdots)
\end{eqnarray}

In the following of this section, detailed analysis for both cases will be described and the worst-case delay of the olMult will be discussed.
\subsection{Worst-case Analysis in On-line Multiplier}
From (\ref{Eq:OnlineMult}) several observations can be made under the assumption that all signals are reset to $0$ initially. 

%\newtheorem{Ob}{Observation}\label{Ob:Ob1}
\begin{Ob}\label{Ob:Ob1}
    The two integer bits of $W[j]$ and $2P[j+1]$ are either $00$ or $11$.
\end{Ob}

This observation can be justified by contradiction. All the combinations of $\widehat{W[j]}$ and the corresponding $z_j$, the most significant 3 bits of $P[j+1]$ and the 2 integer bits of $2P[j+1]$ are listed in Table~\ref{Tab:Observation1}. For $j=-3$, we have $W[-3]=2^{-3}(y_1X[-3])$ according to (\ref{Eq:OnlineMult}). Hence $\widehat{W[-3]}$ is either $11.1$ or $00.0$, with the corresponding $2P[-2]$ being $00$ or $11$ which is the propagation input for the next stage. For $j>-3$, the 3 MSBs of $2^{-3}(x_{j+4}Y[j+1]+y_{j+4}X[j])$ in (eq.xxx) is either $00.0$ or $11.1$ due to the shift of binary point. Also as seen in Table.xxx, if the two integer bits of $\widehat{W[j]}$ are identical, the integer bits of $2P[j+1]$ will be the same. Therefore the propagation of $2P[j+1]$ will not generate diverse bits in the integer part of $W[j]$.
%
\begin{table}[htbp]
\caption{All the combinations of $\widehat{W[j]}$ and the corresponding $z_j$, $P[j+1]$ and $2P[j+1]$.}
\centering
%\vspace{10ex}
\begin{tabular}{c|ccc}
\toprule
 $\widehat{W[j]}$ & $z_j$ & $P[j+1]$ (3 MSBs)  & $2P[j+1]$ (2 MSBs) \\ \midrule
 $00.0$ & $0$ & $00.0$ & $00$\\
 $00.1$ & $1$ & $11.1$ & $11$\\
 $01.0$ & $1$ & $00.0$ & $00$\\
 $01.1$ & $1$ & $00.1$ & $01$\\
 $10.0$ & $\bar{1}$ & $11.0$ & $10$\\
 $10.1$ & $\bar{1}$ & $11.1$ & $11$\\
 $11.0$ & $\bar{1}$ & $00.0$ & $00$\\
 $11.1$ & $0$ & $11.1$ & $11$\\ \bottomrule
\end{tabular}
\label{Tab:Observation1}
\end{table}

\begin{Ob}\label{Ob:Ob2}
    $P[j+1]$ will not chagne if only the integer bits of $W[j]$ change to a new value.
\end{Ob}

According to Observation~\ref{Ob:Ob1}, there are only four possible cases for the change of $W[j]$ as summarized in Table.xxx. This observation describes the situation where a propagation chain annihilates but not necessarily generating a new chain, because $2P[j+1]$ may not maintain its initial value at all times.
%
\begin{table}[htbp]
\caption{Ob2.}
\centering
\begin{tabular}{c|ccc}
\toprule
 $\widehat{W[j]}$ & $z_j$ & $P[j+1]$ (3 MSBs)  & $2P[j+1]$ (2 MSBs) \\ \midrule
 $00.0\rightarrow11.0$ & $0\rightarrow\bar{1}$ & $00.0$ & $00$\\
 $00.1\rightarrow11.1$ & $1\rightarrow\bar{0}$ & $11.1$ & $11$\\
 $11.0\rightarrow00.0$ & $\bar{1}\rightarrow0$ & $00.0$ & $00$\\
 $11.1\rightarrow00.1$ & $0\rightarrow{1}$ & $11.1$ & $11$\\ \bottomrule
\end{tabular}
\label{Tab:Observation2}
\end{table}

% only result in $11.1$ or $00.0$ for the most significant 3 bits of $W[j]$. In addition for $2P[j+1]$, there are only two cases ($01$ and $10$) violate Observation~\ref{Tab:Observation1}. The corresponding value of $\widehat{W[j]}$ are $01.1$ and $10.0$ respectively. 

As stated in Section~\ref{subSec:AnnihilationOfChain}, the annihilation of propagation chain would leads to a reduction of the propagation delay. It is worthwhile exploring the conditions of occurence of Observation~\ref{Ob:Ob2}. For instance, if $x_{-3}\neq0$ and $y_{-3}\neq0$, no more than 6 MSBs of $P[-2]$ will change with respect to the initial value at $t=0$, according to (\ref{Eq:OnlineMult}). Hence maximally 5 MSBs of $2P[-2]$ will change. Besides, at $t=0$ the computation of $2^{-3}(x_{j+4}\cdot Y[j+1]+y_{j+4}\cdot X[j])$ in (\ref{Eq:OnlineMult}) is finished, as assumed previously in the timing model. For $t>0$, the variation of $2P[j]$ propagates from $Stage0$ to $Stage1\cdots Stage4$ with the number of altered MSBs being $5$ to $2$, from $t=\mu$ to $t=4\mu$ respectively. Hence for $Stage4$, we have $Pout(4\mu)=Pout(3\mu)$ according to Observation~\ref{Ob:Ob2}. This stands for an annihilation, and the propagation from $Stage0$ to $Stage5$ is therefore $4\mu$. 

\subsection{Probability of Overclocking Error in On-line Multiplier}

\subsection{Magnitude of Overclocking Error in On-line Multiplier}

%------------------------------------------------------------------------
\section{Probabilistic Model of Truncation Error in On-line Operators}


\end{document} 